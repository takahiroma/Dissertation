\section{Supply Side Analysis: Labor and Taxicabs}
\hspace{0.5cm} Though it is difficult to estimate the driver's labor supply due to lack of panel identifier in the TLC data, there are some incidents or events that might have changed driver's decision of their labor supply. In fact, the days of the least number of trips completed by yellow cabs and also by Uber in 2015 and 2016 were 27th January and 23rd January respectively, when blizzards attacked NYC and the subway was partly closed. The number of trips by yellow cabs decreased by 67\% and 78\% on the days in 2015 and 2016 respectively, compared to daily average of those months, while the equivalents by Uber decreased by 58\% and 62\%. As the data shows only equilibrium, it's impossible to distinguish the cause of the dramatic decrease in trips between demand and supply side. However, while the effect of demand side is ambiguous because people would have ceased to go out due to the arrival of a blizzard, but those out of home would have demanded more taxis and partial closure of the subway might have accelerated it, the supply unambiguously declined. It is interesting that the degree of the decrease is smaller for Uber and one of the main causes would be attributed to Uber’s surge pricing system, which gives incentives to drivers to work during bad conditions. 

Another interest is whether medallions will move to areas where demand temporarily increases expectedly. The following analysis is intended to compare the trips completed by yellow cabs with those by Uber studied in Hall et al.'s paper (2015 \cite{hall2015effects}). They study how the number of active cars changed on a specific date (21st March 2015) in the vicinity of Madison Square Garden, whose capacity is about 20,000 persons and where pop star Ariana Grande had a sold-out concert. Due to high demand, Uber fare hiked by 1.8 times at highest for over an hour, and responding to these high wages, drivers moved to the area and thus supply almost doubled. They also show that average wait time was almost unchanged before and after demand hike; 2.6 minutes. I have checked whether the number of trips by yellow cabs near Madison Square Garden changed during the peak hours (2 hours from hour 22, based on pickup time). The result is that though the number of trips was slightly higher than average during the month, no significant difference can be found.\footnote{The area of Madison Square Garden defined in Hall et al. is somewhat large; 5 avenues long and 15 streets wide. I use narrower area (Location 35 in Figure \ref{fig:Manhattan}) and wider area (Location 24, 34, 35, 37). The result is invariant to the different definitions.} I have also checked that the ratio of the number of trips on that day during 2 hours from hour 22 to other hours remained almost unchanged, compared to the counterparts of other days or locations. These imply that though waiting passengers might have increased, the number of medallions just didn't increase much or longer search time (queuing for passengers) caused by the increase of yellow cabs bounced matching back to the normal level. This might be because 1) drivers can’t expect a long ride as opposed to trips from airports, 2) the taxi stop area near Madison Square Garden is relatively small, 3) Penn station is so close that passengers didn’t try to get on taxis so eagerly, and/or 4) expecting this, drivers didn’t head for Madison Square Garden. This evidence would constitute one of the pieces of justification that I fix the number of medallions in my model.