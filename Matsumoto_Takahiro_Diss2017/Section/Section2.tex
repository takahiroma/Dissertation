\vspace{1.0cm}
\section{Related Literature}
\hspace{0.5cm} The NYC taxi market has been studied by many papers, one of the most seminal and controversial papers being Camerer et al. (1997 \cite{camerer1997labor}) regarding labor supply curve of taxi drivers. Analyzing trip sheets from NYC cab drivers, they conclude that the elasticity of labor supply with respect to wage is negative and insist that this might be due to daily income target and no intertemporally substituting labor and leisure across days\footnote{They call the labor supply decision, "one day at a time" as in the paper title.}, which contradicts with the neoclassical theory. It provoked a lot of discussions and though some studies show the result consistent with the paper like Chou (2002 \cite{chou2002testing}) in the Singapore taxi market, Farber (2005 \cite{farber2005tomorrow}) criticizes that the negative elasticity comes from econometrics issues and insists that it is not actually negative. He indicates that cab drivers’ income effects are small and when to quit on a daily basis is highly dependent on cumulative daily hours to that point. In Farber (2015 \cite{farber2015you}), he uses TLC data with hack/medallion numbers and shows that the elasticity is certainly positive.\footnote{The similar results were shown by using Uber data both in the intensive and extensive margin, though decision making by taxi drivers and Uber drivers might be very different. Chen and Sheldon (2016 \cite{chen2016dynamic}) apply discontinuity design to show the elasticity is positive and causal. Hall and Kruger (2016 \cite{hall2016analysis}) study how Uber drivers operate their sessions with survey data and have found that only 6\% of respondents have some income targets. Sheldon (2015 \cite{sheldon2015income}) replicates the study of Camerer et al. (1997) with Uber data, finding the elasticity being still positive and statistically significant and indicating that measurement errors lead to highly negative elasticities in their paper.}  In my model, I set the number of yellow cabs fixed conditional on locations and dates.

There are also some papers studying demand side or general equilibrium of taxi markets.\footnote{Lagos (2003 \cite{lagos2003analysis}) is a classical paper researching NYC taxi market with a dynamic equilibrium model.} My demand model is mainly based on the two papers; Frechette et al. (2016) and Buchholz (2016 \cite{buchholz2016spatial}), though their main interests lie in the inefficiency of regulations or search frictions in taxi markets, which are different from my purpose of this paper. I somewhat combine the essence of the models in the two papers in a simpler way to study effects of the fare increase of yellow cabs and the rise of ride-hailing services on demand for yellow cabs. Frechette et al. (2016) build demand equations with unobservable wait time and the number of waiting passengers for taxis, with TLC data from 2011 to 2012, taking advantage of the geographical characteristics of Manhattan and using traffic speed outside Manhattan as an instrument (supply shifter), and estimate that the elasticity of demand with respect to wait time is inelastic. I follow the strategy but incorporate spatial aspects in order to take price change into consideration. They also model the labor supply dynamics to study how the equilibrium is influenced by taxi regulations and also matching technologies, which I don't.

Buchholz (2016) studies the spatial equilibrium and search frictions of taxi markets in NYC. He focuses on fare variations depending on distance and estimates demand as a function of price. Importantly, his identification of demand and supply in the spatial equilibrium model crucially depends on fixed labor supply, which he justifies with the notion that medallion limits are binding during the busy time. This is the same strategy I apply in my demand model, though he explicitly admits vacant cabs moving to other locations to search for passengers. He also estimates the effect of the fare increase in 2012 on welfare and studies the value of improved matching technologies and differential pricing used by ride-hailing services, but does not analyze the influence of ride-hailing services on yellow cabs directly.

Cohen et al. (2016 \cite{cohen2016using}) study own price elasticity with Uber data when the price is surged due to tight market conditions. When Uber increases the price, it multiplies standard fare by the surge multiplier shown to passengers. They estimate demand curve, using the property of continuous surge multiplier being rounded to discrete price increments when it is multiplied (e.g. the multiplier 1.249 rounded down to 1.2 times base fare, while 1.250 up to 1.3x), applying the econometric regression discontinuity analysis as well as the propensity score methods and conclude that demand is in most cases inelastic.
