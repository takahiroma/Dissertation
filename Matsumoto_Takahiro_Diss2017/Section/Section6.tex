\vspace{1.0cm}
\section{Concluding Remarks}
\hspace{0.5cm} I have estimated the elasticities of demand for yellow cabs with respect to own price and wait time by constructing the model to connect the relationship between unobservable wait time and the number of waiting passengers with trip data in the NYC taxi market. Both of them are inelastic, which implies that taxis are important public transportation for New Yorkers at least when there were no ride-hailing services. I have also calibrated how Uber, the leader of ride-hailing services, affected the demand for yellow cabs. The estimated result that the elasticity of demand with respect to wait time for yellow cab is larger suggests that Uber is a substitute for yellow cabs to some extent. Uber certainly deprived some passengers of yellow cabs, but it rather successfully generated new demand for door-to-door mobility services. Moreover, Uber's fare cut per se seems to have few impact on decrease of demand for yellow cabs, supporting that the growth of ride-hailing service doesn't come from disruption of taxi markets but rather from new demand hike. Lastly, ride-hailing services have expanded the total consumer surplus to a great extent.

Note that there are some limitations in this study, many of which are due to data constraints. Firstly, I set the number of medallions in each location in Manhattan conditional on hours constant, and thus taxi drivers don't change their labor supply decision both in extensive and intensive manner. The principle of their decision might play a crucial role especially after ride-hailing services became popular and miscellaneous regulation regarding medallions' shifts were abolished. I don't have individual trip data of Uber with respect to dropoff locations, distances or surge pricing. I also don't know which service is provided by Uber, resulting in the assumption that all of the services are operated by UberX. Lack of these information restricts the explanatory power of my model, especially the cross price elasticity for yellow cabs. Much richer data in quality would make it possible to estimate the properties of both supply and demand curves of taxis in a general equilibrium model and the influence of ride-hailing services on the market.


\vspace{1.0cm}
