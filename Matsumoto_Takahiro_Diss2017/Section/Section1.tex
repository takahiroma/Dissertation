\vspace{0.5cm}
\section{Introduction}
\hspace{0.5cm}  This paper aims to study the properties of demand in taxi markets. I estimate the elasticities of demand for taxis with respect to price and wait time and analyze the effects of the increase in taxi fare on social welfare as well as how the rise of ride-hailing services have influenced taxi markets. The markets have many differences compared to other transportation sectors, and one of the most prominent aspects is that both passengers and drivers don't know where they are each other. Moreover, as the fare and the quantity of taxis are highly regulated by authorities in many taxi markets, price and quantity by themselves cannot smoothly vary to be market clearing variables like other markets. Rather, wait time for passengers and search time for drivers vary flexibly to equilibrate supply and demand in the taxi markets, at least in the short run. In this paper, I build a spatial model to surmise these variables from data of the taxi market in New York City (NYC) and then estimate the demand curve.

A challenge to building a demand model is that I don't observe passengers' wait time and also the number of waiting passengers, as the data are just results in equilibrium. Additionally, NYC Taxi and Limousine Commission (TLC), the authority of the market, stopped providing a panel identifier, namely driver's hack numbers and medallion IDs for each trip, due to concern about driver's privacy, so I don't know either each cab's search time or the number of searching cabs. However, as average search time can be calculated from the public data on hourly basis and there are some papers which identify drivers like Frechette et al.(2016 \cite{frechette2016frictions})\footnote{They show that almost all of the yellow cabs are working at least once a day on weekdays and they are intensely used during peak time, for example.}, I handle supply side by fixing the number of working cabs constant conditional on hours and pickup locations with those data and evidence. Under the assumption, I extrapolate the relationship between unobservable wait time and waiting passengers with the observed equilibrium data, the number of searching cabs and so on. Then I estimate the effects of wait time and the fare increase on the number of waiting passengers with two stage least squares (2SLS).

Next, I estimate the effects of Uber's rise on taxi markets using the model specified above. As Uber grew rapidly during 2015 and 2016 but cut its standard fare in NYC by approximately 15 \% in January 2016, it is difficult to tell the difference of the cause of the decrease in the number of trips by yellow cabs between Uber's price cut effects (cross price elasticity) and other reasons. I here assume that the decrease in the number of trips was due to Uber (or broadly, ride-hailing services) under control of wait time and other exogenous factors like weather. I overcome the problem by taking advantage of the existence of minimum fare. Namely, if the fare is smaller than the minimum, there was no price change before and after fare cut, while if it exceeds, the price decreased proportionately. Thereupon, if the number of trips by yellow cabs whose trip distance is long enough so that the minimum fare wouldn't have applied if passengers would have chosen Uber declined sharply than others, then the difference would be accrued to the price change.

I choose the market in NYC as a case study to learn the properties of the demand curve in taxi markets because the city discloses rich individual trip data completed by taxis and ride-hailing services. In addition, TLC increased the fare and thus made it possible to estimate own price elasticity of demand for yellow cabs. Furthermore, Uber cut its standard fare, so I can also study the cross price elasticity. These natural experiments are exploited to study the characteristics of the demand curve.

From the model and the data in the NYC taxi market, I find that the elasticities of demand for taxis with respect to both wait time and price are inelastic, price elasticities are overall independent of the distances of trips, and that the fare increase had a negative impact on the total social welfare. Additionally, I show that the wait time elasticity has become less inelastic after Uber became well-known, which indicates that it is partially recognized as the substitute to taxis, and that Uber's rise has expanded the total demand for door-to-door transportation and thus greatly increased the total consumer surplus.

The structure of this paper is as follows. Section 2 introduces the related literature of demand/supply models of taxi markets and Uber, especially the two main references on which my model is based. Section 3 looks over the taxi/ride-hailing markets in NYC, shows the TLC data, and analyzes the data to find some evidence of those markets. Section 4 builds the demand model by specifying the relationship between wait time and waiting passengers. Then I estimate the model with data, analyze the effects of the fare increase, measure consumer surplus, and discuss the results. Section 5 extends the estimation to see the Uber's effects on the taxi market in NYC. I conclude the study in Section 6.
