\begin{center}
\large Abstract
\end{center}

\indent In this paper, I study the properties of the demand curve of the taxi market in New York City with rich trip data of yellow cabs and ride-hailing services. As the pricing scheme is determined by the authority in the taxi market and thus price does not vary depending on market conditions, I construct a model connecting unobservable waiting passengers and wait time, which would be a part of market clearing variables. Taking advantage of price increase in 2012, I estimate elasticities of demand with respect to price and wait time and have found that they are both inelastic and independent of trip distances. In addition, the elasticity of demand for yellow cabs with respect to wait time after Uber became popular is higher, implying that yellow cabs and ride-hailing services are substitutes each other to some extent. Lastly, welfare analysis suggests that Uber and other ride-hailing services have dramatically increased the total consumer surplus by door-to-door transportation.



\vfill

\begin{center}
\large Acknowledgements
\end{center}

\indent I would like to thank Dr. Nikita Roketskiy for his support throughout my work on this dissertation. I also would like to thank my family and friends who made me who I am and helped me in all of my life.


\vspace{5.0cm}
\hspace{13cm}{\small{Word Count: 9,473}}